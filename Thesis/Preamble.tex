\documentclass[12pt,letterpaper]{report}

\usepackage[hmargin={1.5in,1in},vmargin={1in,1in},marginparsep=0in,marginparwidth=0pt,pdftex]{geometry}
%\usepackage[left=1.5in, right=1in, top=1in, bottom=1in, pdftex]{geometry}
%\usepackage[left=1.5in, pdftex]{geometry}


\usepackage{amsmath}
\usepackage{amsfonts}
\usepackage{amssymb}
\usepackage{graphicx}
\usepackage{subfigure}
\usepackage{setspace}
\usepackage{tabularx}
\usepackage{multirow}
\usepackage{xspace}
\usepackage{url}
\usepackage{algorithm}
\usepackage{algorithmic}
\usepackage{verbatim}
\usepackage{multicol}

\usepackage[small]{caption}

\pdfpagewidth 8.5in
\pdfpageheight 11in
%\topmargin 0in
%\headheight 0in
%\headsep 0in
%\textheight 9in
%\textwidth 6in
%\oddsidemargin 0.5in
%\evensidemargin 0.5in
%\headsep 0in



\graphicspath{{../Images/}}



\newcommand{\SWEEP}{SWEEP\xspace}
\newcommand{\SWEEPexp}{SWarm Experimentation and Evaluation Platform\xspace}
\newcommand{\SWEEPexpbf}{\textbf{SW}arm \textbf{E}xperimentation and \textbf{E}valuation \textbf{P}latform\xspace}

\newcommand{\ECS}{ECS\xspace}
\newcommand{\ECSexp}{\textbf{E}volutionary \textbf{C}omputing for \textbf{S}warms\xspace}

\newcommand{\mutation}[1]{\textit{#1\/}}

\newcommand{\italic}[1]{\textit{#1\/}}
%\newcommand{\bold}[1]{\textbf{#1}}

\newcommand{\ie}{i.e.,\xspace} % in other words
\newcommand{\eg}{e.g.,\xspace} % for example
\newcommand{\apriori}{\textit{a~priori}\xspace}

%%
%% MiKo: macros for displaying Java terms
%%
%%% \/ at the end of a textit{} is the italics correction factor
\newcommand{\jclass}[1]{\texttt{#1\/}}
\newcommand{\jmethod}[1]{\textit{\textbf{#1}\/}}
\newcommand{\true}{\texttt{true}\xspace}
\newcommand{\false}{\texttt{false}\xspace}

%% qw - Quote word, ala perl
\newcommand{\qw}[1]{``#1''}

%
% LATER (make or find)
% * make macros for creating the labels, then mod the \ref's to not need the ch: or fig:, etc
% * marcos for big-O, little-O, etc
% 

%\newcommand{\refFigure}[1]{\textbf{Figure~\ref{fig:#1}}}
%\newcommand{\refFigures}[2]{\textbf{Figures~\ref{fig:#1}--\ref{fig:#2}}}
%\newcommand{\refTable}[1]{\textbf{Table~\ref{tab:#1}}}

\newcommand{\refFigure} [1]{Figure~\ref{fig:#1}}
\newcommand{\refFigures}[2]{Figures~\ref{fig:#1}--\ref{fig:#2}}
\newcommand{\refTable}  [1]{Table~\ref{tab:#1}}

\newcommand{\refAppendix}[1]{Appendix~\ref{app:#1}}
\newcommand{\refChapter}[1]{Chapter~\ref{ch:#1}}
\newcommand{\refSection}[1]{Section~\ref{sec:#1}}
\newcommand{\refAlgorithm}[1]{Algorithm~\ref{alg:#1}}


\newcommand{\note}[1]{\texttt{\textbf{NOTE\@: #1}}}

\newcommand{\placeholder}[1]{\begin{center}\fbox{#1}\end{center}}

%{\Large{$\star$}
%\newcommand{\msb}[1]{\marginpar{\fbox{\footnotesize #1}}}

% http://blue.chem.psu.edu/~rajarshi/misc/tipsntricks.html#q30
\newcommand{\msb}[1]{\marginpar{%
      \vskip-\baselineskip{}
      \raggedright\footnotesize
      \itshape\hrule\smallskip#1\par\smallskip\hrule}} 
      
%\setfiguredirectory{DIRECTORY}

\doublespacing{}
